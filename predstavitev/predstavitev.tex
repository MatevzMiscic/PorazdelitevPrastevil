\documentclass[a4paper,12pt]{article}
\usepackage[slovene]{babel}
\usepackage[utf8]{inputenc}
\usepackage[T1]{fontenc}
\usepackage[a4paper, total={16cm, 22cm}]{geometry}
\usepackage{lmodern}
\usepackage{amsmath,amsfonts}
\usepackage{amsthm}
\usepackage{setspace}

\def\N{\mathbb{N}}
\def\Z{\mathbb{Z}}
\def\Q{\mathbb{Q}}
\def\R{\mathbb{R}}

\theoremstyle{definition}
\newtheorem{definicija}{Definicija}
\newtheorem{zgled}{Zgled}
\theoremstyle{plain}
\newtheorem{izrek}{Izrek}
\newtheorem{lema}{Lema}
\newtheorem{trditev}{Trditev}
\newtheorem{posledica}{Posledica}

\newenvironment{dokaz}{\begin{proof}[\bfseries\upshape\proofname]}{\end{proof}}

\newcommand{\geslo}[2]{\noindent \textbf{#1} \quad #2 \hfill \break}

\setstretch{1.2}

\title{Porazdelitev praštevil}
\author{Matevž Miščič}
\date{22. avgust 2023}



\begin{document}

\maketitle{}

\section{Uvod}
Praštevila so v matematiki pomembna, saj so gradniki vseh naravnih števil. Velja namreč, da lahko vsako naravno število, večje od $1$, zapišemo kot produkt praštevil. Poleg matematike so praštevila pomembna tudi v računalništvu, še posebej v kriptografiji. V tej predstavitvi si bomo ogledali nekaj zanimivih rezultatov o porazdelitvi praštevil.



\section{Praštevilo}

Začnimo z definicijo.

\begin{definicija}
    Praštevilo je naravno število, ki ima natanko $2$ delitelja.
\end{definicija}
Prvih nekaj praštevil je $2, 3, 5, 7, 11, 13, \ldots$.

Naslednja lastnost praštevil je ključnega pomena.

\begin{trditev}
    \label{factorisation}
    Vsako naravno število, večje od $1$, se da zapisati kot produkt števil.
\end{trditev}
\begin{dokaz}
    Trditev bomo pokazali z indukcijo. Naj bo $n$ naravno število, ki je večje od $1$. Če sta $1$ in $n$ edina delitelja števila $n$, je $n$ praštevilo in ni kaj dokazati. Sice obstaja nek delitelj $a$ števila $n$, da velja $1 < a < n$. Torej je $n = ab$ za nek $b \in \N$. Ker je $1 < a, b < n$ lahko po indukcijski predpostavki $a$ in $b$ zapišemo kot produkt praštevil, zato enako velja za njun produkt $n$.
\end{dokaz}

Posledice te trditve je, da je praštevil neskončno mnogo. To je rezultat, ki ga je poznal že Evklid okoli leta 300 pred našim štetjem.

\begin{posledica}
    Praštevil je neskončno mnogo.
\end{posledica}
\begin{dokaz}
    Recimo, da jih je le končno mnogo. Naj bodo $p_1, p_2, \ldots, p_k$ vsa praštevila. Vzemimo $n = p_1p_2 \cdots p_k + 1$. Po prejšnji trditvi vemo, da lahko $n$ zapišemo kot produkt praštevil, ampak očitno nobeno od praštevil ne deli $n$, torej smo prišli v protislovje.
\end{dokaz}



\begin{thebibliography}{1}
    \bibitem{1}
    J.~A.~Hocutt in P.~L.~Robinson, \emph{Everywhere Differentiable, Nowhere Continuous Functions}, Amer.~Math.~Monthly \textbf{125} (2018) 923--928.
    \bibitem{2}
    P.~R.~Halmos, \emph{Problems for Mathematicians, Young and Old}, Dolciani Mathematical Expositions \textbf{12}, Mathematical Association of America, Washington, 1991.
\end{thebibliography}


\end{document}